\documentclass[
	ngerman,
	toc=listof, % Abbildungsverzeichnis sowie Tabellenverzeichnis in das Inhaltsverzeichnis aufnehmen
	toc=bibliography, % Literaturverzeichnis in das Inhaltsverzeichnis aufnehmen
	footnotes=multiple, % Trennen von direkt aufeinander folgenden Fußnoten
	parskip=half, % vertikalen Abstand zwischen Absätzen verwenden anstatt horizontale Einrückung von Folgeabsätzen
	numbers=noendperiod % Den letzten Punkt nach einer Nummerierung entfernen (nach DIN 5008)
]{scrartcl}
\PassOptionsToPackage{automark, headsepline=.4pt, plainheadsepline}{scrlayer-scrpage}

\pdfminorversion=5 % erlaubt das Einfügen von pdf-Dateien bis Version 1.7, ohne eine Fehlermeldung zu werfen (keine Garantie für fehlerfreies Einbetten!)
\usepackage[utf8]{inputenc} % muss als erstes eingebunden werden, da Meta/Packages ggfs. Sonderzeichen enthalten

\newcommand{\titel}{Entwicklung einer allgemeinen Ressourcenverwaltung}
\newcommand{\untertitel}{für Diagnostikgeräte via gRPC}

\newcommand{\autorName}{Lukas Klettke}
\newcommand{\autorAnschrift}{Am Mühlenteich 17}
\newcommand{\autorOrt}{23611 Bad Schwartau}
\newcommand{\autorPfnr}{Prüflingsnummer: 08 15}

\newcommand{\betriebLogo}{LogoBetrieb.pdf}
\newcommand{\betriebName}{\textsc{EUROIMMUN} Medizinische Labordiagnostika AG}
\newcommand{\betriebAnschrift}{Am Seekamp 31}
\newcommand{\betriebOrt}{23560 Lübeck}

\newcommand{\ausbildungsberuf}{Fachinformatiker für Anwendungsentwicklung}
\newcommand{\betreff}{Dokumentation zur betrieblichen Projektarbeit}
\newcommand{\pruefungstermin}{Winter 2021}
\newcommand{\abgabeOrt}{Lübeck}
\newcommand{\abgabeTermin}{13.05.2019}
 % Metadaten zu diesem Dokument (Autor usw.)
% Anpassung an Landessprache ---------------------------------------------------
\usepackage[ngerman]{babel}

% Umlaute ----------------------------------------------------------------------
%   Umlaute/Sonderzeichen wie äüöß direkt im Quelltext verwenden (CodePage).
%   Erlaubt automatische Trennung von Worten mit Umlauten.
% ------------------------------------------------------------------------------
\usepackage[T1]{fontenc}
\usepackage{textcomp} % Euro-Zeichen etc.

% Schrift ----------------------------------------------------------------------
\usepackage[pdfspacing]{classicthesis}
\usepackage{lmodern} % bessere Fonts
\usepackage{relsize} % Schriftgröße relativ festlegen

% Tabellen ---------------------------------------------------------------------
\usepackage{xcolor}
\usepackage{colortbl}
\usepackage{tabularx}
% für lange Tabellen
\usepackage{longtable}
\usepackage{array}
\usepackage{ragged2e}
\usepackage{lscape}

% Einfache Definition der Zeilenabstände und Seitenränder etc.
\usepackage{setspace}
\usepackage{geometry}

% Grafiken ---------------------------------------------------------------------
\usepackage[dvips,final]{graphicx} % Einbinden von JPG-Grafiken ermöglichen
\usepackage{smartdiagram}
\usepackage{graphics} % keepaspectratio
\usepackage{graphicx}
\usepackage{floatflt} % zum Umfließen von Bildern
\graphicspath{{Bilder/}} % hier liegen die Bilder des Dokuments

% Sonstiges --------------------------------------------------------------------
\usepackage[titles]{tocloft} % Inhaltsverzeichnis DIN 5008 gerecht einrücken
\usepackage{amsmath,amsfonts} % Befehle aus AMSTeX für mathematische Symbole
\usepackage{enumitem} % anpassbare Enumerates/Itemizes
\usepackage{xspace} % sorgt dafür, dass Leerzeichen hinter parameterlosen Makros nicht als Makroendezeichen interpretiert werden

\usepackage{makeidx} % für Index-Ausgabe mit \printindex
\usepackage[printonlyused]{acronym} % es werden nur benutzte Definitionen aufgelistet
\usepackage{nomencl}

% zum Einbinden von Programmcode -----------------------------------------------
\usepackage{listings}
\usepackage{hyperref} %For adding hyperlinks

\usepackage{protobuf/lang}  % include language definition for protobuf
\usepackage{protobuf/style} % include custom style for proto declarations.

\definecolor{hellgelb}{rgb}{1,1,0.9}
\definecolor{colKeys}{rgb}{0,0,1}
\definecolor{colIdentifier}{rgb}{0,0,0}
\definecolor{colComments}{rgb}{0,0.5,0}
\definecolor{colString}{rgb}{1,0,0}
\lstset{
	float=hbp,
	basicstyle=\footnotesize,
	identifierstyle=\color{colIdentifier},
	keywordstyle=\color{colKeys},
	stringstyle=\color{colString},
	commentstyle=\color{colComments},
	backgroundcolor=\color{hellgelb},
	columns=flexible,
	tabsize=2,
	frame=single,
	extendedchars=true,
	showspaces=false,
	showstringspaces=false,
	numbers=left,
	numberstyle=\tiny,
	breaklines=true,
	breakautoindent=true,
	captionpos=b,
}
\lstdefinelanguage{cs}{
	sensitive=false,
	morecomment=[l]{//},
	morecomment=[s]{/*}{*/},
	morestring=[b]",
	morekeywords={
		abstract,event,new,struct,as,explicit,null,switch
		base,extern,object,this,bool,false,operator,throw,
		break,finally,out,true,byte,fixed,override,try,
		case,float,params,typeof,catch,for,private,uint,
		char,foreach,protected,ulong,checked,goto,public,unchecked,
		class,if,readonly,unsafe,const,implicit,ref,ushort,
		continue,in,return,using,decimal,int,sbyte,virtual,
		default,interface,sealed,volatile,delegate,internal,short,void,
		do,is,sizeof,while,double,lock,stackalloc,
		else,long,static,enum,namespace,string},
}
\lstdefinelanguage{natural}{
	sensitive=false,
	morecomment=[l]{/*},
	morestring=[b]",
	morestring=[b]',
	alsodigit={-,*},
	morekeywords={
		DEFINE,DATA,LOCAL,END-DEFINE,WRITE,CALLNAT,PARAMETER,USING,
		IF,NOT,END-IF,ON,*ERROR-NR,ERROR,END-ERROR,ESCAPE,ROUTINE,
		PERFORM,SUBROUTINE,END-SUBROUTINE,CONST,END-FOR,END,FOR,RESIZE,
		ARRAY,TO,BY,VALUE,RESET,COMPRESS,INTO,EQ},
}
\lstdefinelanguage{php}{
	sensitive=false,
	morecomment=[l]{/*},
	morestring=[b]",
	morestring=[b]',
	alsodigit={-,*},
	morekeywords={
		abstract,and,array,as,break,case,catch,cfunction,class,clone,const,
		continue,declare,default,do,else,elseif,enddeclare,endfor,endforeach,
		endif,endswitch,endwhile,extends,final,for,foreach,function,global,
		goto,if,implements,interface,instanceof,namespace,new,old_function,or,
		private,protected,public,static,switch,throw,try,use,var,while,xor
		die,echo,empty,exit,eval,include,include_once,isset,list,require,
		require_once,return,print,unset},
} % verwendete Packages
% !TEX root = ../Projektdokumentation.tex

% Seitenränder -----------------------------------------------------------------
\setlength{\topskip}{\ht\strutbox} % behebt Warnung von geometry
\geometry{a4paper,left=20mm,right=20mm,top=25mm,bottom=35mm}

\usepackage[
	automark, % Kapitelangaben in Kopfzeile automatisch erstellen
	headsepline=.4pt,
	plainheadsepline
]{scrlayer-scrpage}

% Kopf- und Fußzeilen ----------------------------------------------------------
\pagestyle{scrheadings}

% Kopfzeile
\renewcommand{\headfont}{\normalfont} % Schriftform der Kopfzeile
\ihead{\large{\textsc{\titel}}\\ \small{\untertitel} \\[2ex] \textit{\headmark}}
\chead{}
\ohead{\includegraphics[scale=1.05]{\betriebLogo}}
\setlength{\headheight}{10mm} % Höhe der Kopfzeile
%\setheadwidth[0pt]{textwithmarginpar} % Kopfzeile über den Text hinaus verbreitern (falls Logo den Text überdeckt)

% Fußzeile
\ifoot{\autorName}
\cfoot{}
\ofoot{\pagemark}

\newcommand{\headingSpace}{1.5cm}

\renewcommand*{\othersectionlevelsformat}[3]{
  \makebox[\headingSpace][l]{#3\autodot}
}

\cftsetindents{section}{0.0cm}{\headingSpace}
\cftsetindents{subsection}{0.0cm}{\headingSpace}
\cftsetindents{subsubsection}{0.0cm}{\headingSpace}
\cftsetindents{figure}{0.0cm}{\headingSpace}
\cftsetindents{table}{0.0cm}{\headingSpace}

\onehalfspacing % Zeilenabstand 1,5 Zeilen
\frenchspacing % erzeugt ein wenig mehr Platz hinter einem Punkt

\clubpenalty = 10000
\widowpenalty = 10000
\displaywidowpenalty = 10000

\counterwithout{footnote}{section} % Fußnoten fortlaufend durchnummerieren
\setcounter{tocdepth}{3} % im Inhaltsverzeichnis werden die Kapitel bis zum Level der subsubsection übernommen
\setcounter{secnumdepth}{3} % Kapitel bis zum Level der subsubsection werden nummeriert

% Aufzählungen anpassen
\renewcommand{\labelenumi}{\arabic{enumi}.}
\renewcommand{\labelenumii}{\arabic{enumi}.\arabic{enumii}.}
\renewcommand{\labelenumiii}{\arabic{enumi}.\arabic{enumii}.\arabic{enumiii}}

% Tabellenfärbung:
\definecolor{heading}{rgb}{0.64,0.78,0.86}
\definecolor{odd}{rgb}{0.9,0.9,0.9} % Definitionen zum Aussehen der Seiten
\input{Allgemein/Befehle} % eigene allgemeine Befehle, die z.B. die Arbeit mit LaTeX erleichtern

\begin{document}

\phantomsection
\thispagestyle{plain}
\pdfbookmark[1]{Deckblatt}{deckblatt}
\begin{titlepage}
	\centering
	{\scshape\LARGE Abschlussprüfung \pruefungstermin \par}
	\vspace{1cm}
	{\scshape\LARGE \ausbildungsberuf \par}
	\vspace{1cm}
	{\scshape\Large Dokumentation zur betrieblichen Projektarbeit \par}
	\vspace{1cm}
	{\huge\bfseries\titel \par}
	{\scshape\Large \untertitel \par}
	\vspace{2cm}
	{\scshape\Large Prüfungsbewerber: \par}
	{\Large\itshape \autorName \par}
	{\Large\itshape \autorAnschrift \par}
	{\Large\itshape \autorOrt \par}
	\vfill

	% Bottom of the page
	{\large \today\par}
\end{titlepage}
\cleardoublepage

\phantomsection
\thispagestyle{empty}
\pdfbookmark[1]{Eidesstattliche Erklärung}{ihkdeckblatt}
\cleardoublepage

% Preface --------------------------------------------------------------------
\renewcommand{\contentsname}{Inhaltsverzeichnis}
\phantomsection
\pagenumbering{Roman}
\pdfbookmark[1]{Inhaltsverzeichnis}{inhalt}
\tableofcontents
\cleardoublepage

\pagenumbering{arabic}
% !TEX root = Projektdokumentation.tex
 % !TEX root = ../Projektdokumentation.tex
\section{Einleitung}
\label{sec:Einleitung}

\subsection{Vorstellung der eigenen Person} 
\label{sec:eigene Person}
Mein Name ist Lukas Klettke. Ich bin am 14.01.2001 in Lübeck geboren und in Bad Schwartau aufgewachsen. Dort habe ich die Grundschule und das Leibniz Gymnasium besucht. Nach zwölf Jahren Schulzeit habe ich meinen Schulweg im Jahre 2019 mit dem Abitur abgeschlossen.

Direkt nach Abschluss der Schule habe ich im August 2019 eine Ausbildung zum Fachinformatiker für Anwendungsentwicklung begonnen und bin in der Softwareentwicklung für Diagnostikgeräte tätig.

In meiner Freizeit bin ich als ehrenamtlicher Schwimmtrainer tätig, segle und fahre Rennrad.

\subsection{Vorstellung des Ausbildungsbetriebs} 
\label{sec:Ausbildungsbetrieb}
Mein Ausbildungsbetrieb ist die {\betriebName} mit Sitz in {\betriebOrt} und Zweigstellen in Groß Grönau, Selmsdorf und Dassow im Norden und Rennersdorf, Pegnitz und Bernstadt im Süden Deutschlands. Durch den Verkauf der Firma im Dezember 2017 befindet sich {\betriebNameKzf} in Besitz von {\mutterBetriebName}, einem US-amerikanischen Technologieunternehmen im Bereich der Chemie- und Medizintechnik.

{\betriebNameKzf} ist ein Hersteller für diverse medizinische Diagnostika von Autoimmun-, Infektionskrankheiten und Allergien, aber auch im Bereich der Automatisierung. Die Ausbildung findet in Dassow in der Forschung und Entwicklung von Software zur Steuerung von Diagnostikautomaten statt.

Insgesamt hat {\betriebNameKzf} mehr als 3.200 Mitarbeiter in 17 Ländern.

\subsection{Projektauslöser} 
\label{sec:Projektauslöser}
Neben der Herstellung von medizinischen Diagnostika zur manuellen Durchführung, werden Geräte zur automatisierten Durchführung dessen hergestellt und vertrieben. Diese Diagnostikautomaten arbeiten mit diversen unterschiedlichen Betriebsmitteln (z.B. Reinigungsflüssigkeit zur Reinigung der Schläuche, Probenträger, etc.), welche ebenfalls von {\betriebNameKzf} an die Kunden verkauft werden.

Anhand der verbrauchten Betriebsmittel der Geräte, wird die Menge, der in Zukunft benötigten, berechnet und die Preise dementsprechend auf den Kunden angepasst. Außerdem wird je nach Verbrauch der Labore die Produktions- und Lagermenge optimiert.

Derzeit werden jedoch keine Daten der Verbräuche von den Gerätesoftwares erhoben, was eine manuelle Berechnung derer zur Folge hat. Diese Berechnung wird durch Außendienstmitarbeiter durchgeführt, welche die Labore besuchen und die durchgeführten Testmengen als Maßstab nutzen. Je nach Art des Gerätebetriebs ist der Verbrauch jedoch unterschiedlich: Werden z.B. 500 Tests am Stück durchgeführt, ist das Verhalten des Geräts ein Anderes, als wenn über eine Zeit von zwei Wochen 500 Tests durchgeführt werden. Somit ist bei der Berechnung eine gewisse Ungenauigkeit vorhanden, die in Zusammenhang mit einer großen Anzahl an Kunden, Differenzen zwischen berechneten und reellen Verbräuchen verursacht. 

Durch eine automatisierte und genauere Berechnung mithilfe von protokollierten Ressourcenverbräuchen könnten Punkte wie Preisgestaltung, Produktionsmenge oder Lagerhaltung weiter optimiert werden und somit Geld einsparen bzw. Gewinn maximieren.

\subsection{Projektumfeld}
\label{sec:Projektumfeld}
Das Projektumfeld ist der {\betriebNameKzf} Standort in Dassow. Dort befindet sich ein Teil der Entwicklung der Diagnostikgeräte und der zugehörigen Software. 

Bei diesem Projekt handelt es sich um eine Software, die ausschließlich intern eingesetzt werden soll.

\subsection{Projektziel}
\label{sec:Projektziel}
Ziel des Projekts ist es eine Schnittstelle zu definieren, die unabhängig von Diagnostikgerät und der entsprechenden Software implementiert werden kann. Durch diese Schnittstelle wird definiert, in welcher Form die Verbrauchsdaten abgefragt und verarbeitet werden.

Anhand dessen wird eine Software geschrieben, die die Ressourcenverbräuche verarbeitet, eine Gesamtberechnung durchführt und den Export einer \glqq .xlsx\grqq \xspace ermöglicht, um die nachstehende Kalkulation mittels Microsoft Excel zu gewährleisten.

Die Planung eines solchen Projekts existiert bereits mehrere Jahre und wurde von der Geschäftsführung in Auftrag gegeben. Das Ziel dessen ist es Daten über die Nutzung der {\betriebNameKzf} Diagnostikgeräte zur erheben, welche zur Analyse von weiteren Optimierungmöglichkeiten dienen.

\subsection{Projektschnittstellen}
\label{sec:Projektschnittstellen}
Das Projekt stellt eine {\acs{gRPC}} Schnittstelle bereit. Diese wird mithilfe einer \glqq .proto\grqq \xspace Datei definiert. Die Schnittstelle wird auf Seite des Clients implementiert und auf Seite des Servers offen gelassen, sodass die unterschiedlichen Softwares der Diagnostikgeräte diese implementieren können und die Freiheit haben, je nach Architektur und Speicherung, die Daten bereitzustellen.

Des Weiteren wird der Export einer \glqq .xlsx\grqq \xspace Datei angeboten.
 % !TEX root = ../Projektdokumentation.tex
\section{Projektplanung}
\label{sec:Projektplanung}

\subsection{Projektphasen} 
\label{sec:Projektphasen}

Tabelle~\ref{tab:Zeitplanung} zeigt die vorgesehenen Phase des Projektes.
\tabelle{Zeitplanung}{tab:Zeitplanung}{Zeitplanung}\\ 

\subsection{Ist-Analyse} 
\label{sec:IstAnalyse}
Zur Zeit besuchen Außendienstmitarbeiter Labore, die Diagnostikgeräte von {\betriebNameKzf} verwenden, um die Ressourcenverbräuche zu berechnen. Um weiterhin genügend Betriebsmittel vorzuhalten, werden retrospektiv die Verbräuche entsprechend zur Zeit berechnet und die weitere Versorgung sicher gestellt. Diese Berechnung geschieht anhand der Anzahl durchschnittlich durchgeführter Tests und Aussagen der Labormitarbeiter, falls absehbar ist, dass zukünftig vom Durchschnitt abgewichen wird.

Diese Arbeitsweise erfordert gut geschulte Mitarbeiter.

\subsection{Soll-Konzept}
\label{sec:SollKonzept}
Durch die neu entwickelte Software soll es den Außendienstmitarbeitern möglich sein die Ressourcenverbräuche genauer zu berechnen, den zukünftigen Bedarf präziser zu planen, somit die Preisgestaltung anzupassen und die Produktion und Lagerhaltung zu optimieren. Der Mitarbeiter kann mit einem Windows PC sich in das lokale Netzwerk des Labors einwählen und über die gegebene Schnittstelle eine Verbindung mit den Computern der Diagnostikautomaten aufbauen. Mithilfe dieser Verbindung werden die Verbräuche der Betriebsmittel abgefragt.

Das Programm soll für die Möglichkeit auf internationale Anwendung in Englisch geschrieben werden und auf einem Windows 7/8/10/11 PC laufen. Die Software wird generisch entwickelt, sodass der zukünftige Ausbau mithilfe anderer Technologien möglich ist. Sowohl die Seite des Diagnostikgeräts, als auch die des Außendienstmitarbeiters werden in C\# entwickelt.

\subsection{\glqq Make or Buy\grqq}
\label{sec:MakeOrBuy}
Da der Anwendungsfall spezifisch für von {\betriebNameKzf} entwickelte Diagnostikgeräte gilt, gab es keine käufliche Software, die den Ansprüchen gerecht wird.

\subsection{Kosten- und Ablaufplanung}
\label{sec:KostenAblaufPlanung}

\subsubsection{Wirtschaftlichkeitsprüfung}
\label{sec:Wirtschaftlichkeitspruefung}
Die Wirtschaftlichkeit wird in den beiden folgenden Punkten genau beschrieben.

\subsubsection{Projektkosten}
\label{sec:Projektkosten}
Die Projektkosten werden mit einigen variablen Parametern betrachtet, da echte Daten seitens der Geschäftsleitung nicht herausgegeben werden. Aus diesem Grund wird mit fiktiven Stundensätzen gearbeitet. Der Stundensatz eines Auszubildenden wird mit 75 EUR und der eines Mitarbeiters mit 100 EUR angesetzt. In diesen Stundensätzen sind neben Gehaltszahlungen Kostenbeiträge  wie Lohnnebenkosten und Sozialbeiträge enthalten. Die im Unternehmen typischen Gemeinkosten, wie Miete, Reinigungskosten der Räumlichkeiten oder Abschreibungen auf das technische Equipment werden zusätzlich mit 15 EUR/Stunde angesetzt. Das Projekt wurde mit 70 Stunden angesetzt, woraus sich ein Gesamtbudget von 6.300 EUR ergibt (s. Tabelle~\ref{tab:Kostenaufstellung}).
\tabelle{Kostenaufstellung}{tab:Kostenaufstellung}{Kostenaufstellung}\\

Kosten für die genutzte Hardware werden nicht angesetzt, da davon ausgegangen wird, dass diese schon vorhanden ist.

\subsubsection{Amortisationsdauer}
\label{sec:Amortisationsdauer}
Die Amortisation des Projekts ist nicht das primäre Ziel. Wie oben genannt dient es zur Analyse von Daten in Bezug auf das Nutzungsverhalten der Kunden und zukünftige Optimierungen. Aus diesem Grund ist keine Amortisationsdauer zu berechnen.

Durch die hohe Anzahl an genutzten Diagnostikgeräten von {\betriebNameKzf} in Laboren auf der ganzen Welt, ist der Gewinnverlust hochgerechnet auf ein Gerät sehr gering. Bei einer Anzahl von beispielsweise 5.000 verkauften Geräten beträgt der Gewinnverlust ca. 1,25 EUR pro Gerät.

\subsection{Qualitätsanforderungen}
\label{sec:Qualitätsanforderungen}
Da die Anwendung unter Anderem die Kosten für Mitarbeiterschulungen senken soll, muss sie einfach und intuitiv zu bedienen sein. Bei geringfügigen Fehlern soll die Software ausfallsicher sein und durch die Abfrage von unbestimmten Größen an Datenmengen soll die Kommunikation performant und schlank sein.

Die Bereitstellung einer frei implementierbaren Schnittstelle verlangt eine gute Dokumentation und Kommentierung des Quellcodes, sodass Entwickler diese Schnittstelle einfach nutzen können. Da die Softwares zur Steuerung der Diagnostikgeräte für Windows entwickelt werden, ist es nötig die Schnittstelle ebenfalls für Windows zu entwickeln und daraus resultierend auch die Software zur Abfrage der Betriebsmittelverbräuche.
 % !TEX root = ../Projektdokumentation.tex
\section{Entwurfsphase}
\label{sec:Entwurfsphase}

\subsection{Zielplattform}
\label{sec:Zielplattform}
Für die 64 Bit Version von Windows 10 wird ein Prozessor mit mindestens 1 GHz Arbeitsleistung oder ein {\acs{SoC}} benötigt. Die Mindestkapazität des RAM liegt bei 2 GB, der Festplattenspeicher muss mindestens 32 GB groß sein. Die Grafikkarte muss über DirectX 9 oder höher mit einem {\acs{WDDM}} 1.0 Treiber verfügen.

\subsection{Qualitätssicherung}
\label{sec:Qualitätssicherung}
Zur Sicherung der Qualität wird die Zuverlässigkeit der Software getestet. Beide Seiten der Schnittstelle müssen zu jeder Zeit erreichbar sein und auf Anfragen reagieren bzw. Anfragen senden können. Gleichzeitig muss in Fehlerfällen reagiert werden und eine weitere Bereitstellung der Dienste gesichert sein. Zur Gewährleistung dessen wird die Software im fertigen Zustand Stresstests mit großen Datenmengen und absichtlich verursachten Fehlerfällen ausgesetzt. 

Zur Gewährleistung der Einsatzbereitschaft ist es erforderlich, die einzubindenden Bibliotheken eindeutig zu dokumentieren um die Menge der möglichen Implementierungsfehler so klein wie möglich zu halten.
 % !TEX root = ../Projektdokumentation.tex
\section{Realisierung}
\label{sec:Realisierung}

\subsection{Eingesetzte Technologien}
\label{sec:EingesetzteTechnologien}
Für die Entwicklung der Schnittstelle mit der die Verbräuche der Betriebsmittel der Diagnostikgeräte abgefragt werden, wird {\acs{gRPC}} verwendet. {\acs{gRPC}} ist ein Open-Source {\acs{RPC}} System, welches von Google entwickelt wird.

{\ac{RPC}} ist eine Technologie, die es möglich macht, Prozeduren auf anderen Geräte auszuführen, als auf dem, wo es aufgerufen wird (meistens innerhalb eines Netzwerks). Durch diese Auslagerung kann Datenverarbeitung auf andere Geräte ausgelagert werden, ohne dass ein Unterschied in der Entwicklung entsteht. Daraus ergibt sich eine Form der Client-Server-Architektur, bei der der aufrufende Part den Client und der ausführende Part den Server darstellt. Die Kommunikation basiert auf dem \glqq request-response\grqq \space Protokoll, welche synchron via http abläuft. Schickt der Client eine Abfrage, ist er während der Bearbeitung durch den Server blockiert\footnote{Vgl. Wikipedia: \url{https://en.wikipedia.org/wiki/Remote_procedure_call} (Stand 08.11.2021 10:15 Uhr)}.

Die Wahl der Technologie fiel auf {\acs{gRPC}}, da dieses bereits in der Firma genutzt wurde und somit eine Vorgabe darstellte.

Die Anwendung wird mit C\# entwickelt. Dafür wird die Version 4.8 des .NET Frameworks verwendet.

Zum exportieren von \glqq .xlsx\grqq \space Dateien wird das {\acs{NuGet}}-Paket \glqq Microsoft.Office.Interop.Excel\grqq \space verwendet. Die Dokumentation befindet sich auf der Microsoft Docs Webseite\footnote{Microsoft.Office.Interop.Excel: \url{https://docs.microsoft.com/en-us/dotnet/api/microsoft.office.interop.excel?view=excel-pia}} des Frameworks.

Für die graphische Benutzeroberfläche wird die {\betriebNameKzf} interne {\acs{Material Design}} Bibliothek verwendet.

Zur Strukturierung der Software-Architektur wird {\acs{Prism}} eingesetzt. Durch die Nutzung ergibt sich eine einfach Verwendung von  {\acs{IoC}}, die die Struktur der Software deutlich übersichtlicher gestalten lässt.

\end{document}