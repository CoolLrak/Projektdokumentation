% !TEX root = ../Projektdokumentation.tex
\section{Einleitung}
\label{sec:Einleitung}

\subsection{Vorstellung der eigenen Person} 
\label{sec:eigene Person}
Mein Name ist Lukas Klettke. Ich bin am 14.01.2001 in Lübeck geboren und in Bad Schwartau aufgewachsen. Dort habe ich die Grundschule und das Leibniz Gymnasium besucht. Nach zwölf Jahren Schulzeit habe ich meinen Schulweg im Jahre 2019 mit dem Abitur abgeschlossen.

Direkt nach Abschluss der Schule habe ich im August 2019 eine Ausbildung zum Fachinformatiker für Anwendungsentwicklung begonnen und bin in der Softwareentwicklung für Diagnostikgeräte tätig.

In meiner Freizeit bin ich als ehrenamtlicher Schwimmtrainer tätig, segle und fahre Rennrad.

\subsection{Vorstellung des Ausbildungsbetriebs} 
\label{sec:Ausbildungsbetrieb}
Mein Ausbildungsbetrieb ist die {\betriebName} mit Sitz in {\betriebOrt} und Zweigstellen in Groß Grönau, Selmsdorf und Dassow im Norden und Rennersdorf, Pegnitz und Bernstadt im Süden Deutschlands. Durch den Verkauf der Firma im Dezember 2017 befindet sich {\betriebNameKzf} in Besitz von {\mutterBetriebName}, einem US-amerikanischen Technologieunternehmen im Bereich der Chemie- und Medizintechnik.

{\betriebNameKzf} ist ein Hersteller für diverse medizinische Diagnostika von Autoimmun-, Infektionskrankheiten und Allergien, aber auch im Bereich der Automatisierung. Die Ausbildung findet in Dassow in der Forschung und Entwicklung von Software zur Steuerung von Diagnostikautomaten statt.

Insgesamt hat {\betriebNameKzf} mehr als 3.200 Mitarbeiter in 17 Ländern.

\subsection{Projektauslöser} 
\label{sec:Projektauslöser}
Durch den Vertrieb diverser Automatisierungslösungen zur Diagnostik, ist ein weiteres Standbein von {\betriebNameKzf} der Verkauf der Betriebsmittel für die Diagnostikgeräte. Zur Zeit ist es nicht möglich die reellen Resourcenverbräuche der Diagnostikgeräte der Kunden zu überwachen, da es keine Software gibt, die die Ressourcenverbräuche abfragen kann. Die aktuelle Lösung bedarf der manuellen Bearbeitung mithilfe von Außendienstmitarbeitern, die die Labore besuchen. 

Da dies sehr zeit- und kostenintensiv ist, soll eine Software entwickelt werden, die die Ressourcen der Geräte abfragen und graphisch aufarbeiten kann. Es soll eine Schnittstelle bereit gestellt werden, die von den unterschiedlichen Gerätesoftwares implementiert werden kann, sodass über ein zentrales System die gesamten Ressourcenverbräuche innerhalb eines Labors gleichzeitig abgefragt werden können.