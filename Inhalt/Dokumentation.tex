% !TEX root = ../Projektdokumentation.tex
\section{Dokumentation}
\label{sec:Dokumentation}

\subsection{Dokumentation der Software}
\label{sec:DokumentationSoftware}
Zur Weiterentwicklung durch andere Entwickler wurde der Quellcode der Software mit Kommentaren gekennzeichnet. Neben einer kurzen Beschreibung der Klassen und Funktionen wurde darauf geachtet, dass Variablen möglichst selbsterklärend benannt sind.

Des weiteren wurde eine Anleitung zur Nutzung der Software geschrieben, die einen kurzen Überblick für den Anwender gibt.

\subsection{Dokumentation der Schnittstelle}
\label{sec:DokumentationSchnittstelle}
Zur Einbindung der Schnittstelle in andere Softwares zur Steuerung von Diagnostikgeräten, wurde ein {\acs{NuGet}}-Paket erstellt, welches implementiert werden muss. Hierbei wurde ebenfalls darauf geachtet, dass Variablennamen selbsterklärend sind. Außerdem wurden die Klassen und Funktionen kommentiert und beschreiben Inhalt und Rückgabeparameter. Des weiteren wurde zum {\acs{NuGet}}-Paket eine kurze Beschreibung hinzugefügt.