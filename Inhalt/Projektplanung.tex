% !TEX root = ../Projektdokumentation.tex
\section{Projektplanung}
\label{sec:Projektplanung}

\subsection{Projektphasen} 
\label{sec:Projektphasen}

Tabelle~\ref{tab:Zeitplanung} zeigt die vorgesehenen Phase des Projektes.
\tabelle{Zeitplanung}{tab:Zeitplanung}{Zeitplanung}\\ 

\subsection{Ist-Analyse} 
\label{sec:IstAnalyse}
Zur Zeit besuchen Außendienstmitarbeiter Labore, die Diagnostikgeräte von {\betriebNameKzf} verwenden, um die Ressourcenverbräuche zu berechnen. Um weiterhin genügend Betriebsmittel vorzuhalten, werden retrospektiv die Verbräuche entsprechend zur Zeit berechnet und die weitere Versorgung sicher gestellt. Diese Berechnung geschieht anhand der Anzahl durchschnittlich durchgeführter Tests und Aussagen der Labormitarbeiter, falls absehbar ist, dass zukünftig vom Durchschnitt abgewichen wird.

Diese Arbeitsweise erfordert gut geschulte Mitarbeiter.

\subsection{Soll-Konzept}
\label{sec:SollKonzept}
Durch die neu entwickelte Software soll es den Außendienstmitarbeitern möglich sein die Ressourcenverbräuche genauer zu berechnen, den zukünftigen Bedarf präziser zu planen, somit die Preisgestaltung anzupassen und die Produktion und Lagerhaltung zu optimieren. Der Mitarbeiter kann mit einem Windows PC sich in das lokale Netzwerk des Labors einwählen und über die gegebene Schnittstelle eine Verbindung mit den Computern der Diagnostikautomaten aufbauen. Mithilfe dieser Verbindung werden die Verbräuche der Betriebsmittel abgefragt.

Das Programm soll für die Möglichkeit auf internationale Anwendung in Englisch geschrieben werden und auf einem Windows 7/8/10/11 PC laufen. Die Software wird generisch entwickelt, sodass der zukünftige Ausbau mithilfe anderer Technologien möglich ist. Sowohl die Seite des Diagnostikgeräts, als auch die des Außendienstmitarbeiters werden in C\# entwickelt.

\subsection{\glqq Make or Buy\grqq}
\label{sec:MakeOrBuy}
Da der Anwendungsfall spezifisch für von {\betriebNameKzf} entwickelte Diagnostikgeräte gilt, gab es keine käufliche Software, die den Ansprüchen gerecht wird.

\subsection{Kosten- und Ablaufplanung}
\label{sec:KostenAblaufPlanung}

\subsubsection{Wirtschaftlichkeitsprüfung}
\label{sec:Wirtschaftlichkeitspruefung}
Die Wirtschaftlichkeit wird in den beiden folgenden Punkten genau beschrieben.

\subsubsection{Projektkosten}
\label{sec:Projektkosten}
Die Projektkosten werden mit einigen variablen Parametern betrachtet, da echte Daten seitens der Geschäftsleitung nicht herausgegeben werden. Aus diesem Grund wird mit fiktiven Stundensätzen gearbeitet. Der Stundensatz eines Auszubildenden wird mit 75 EUR und der eines Mitarbeiters mit 100 EUR angesetzt. In diesen Stundensätzen sind neben Gehaltszahlungen Kostenbeiträge  wie Lohnnebenkosten und Sozialbeiträge enthalten. Die im Unternehmen typischen Gemeinkosten, wie Miete, Reinigungskosten der Räumlichkeiten oder Abschreibungen auf das technische Equipment werden zusätzlich mit 15 EUR/Stunde angesetzt. Das Projekt wurde mit 70 Stunden angesetzt, woraus sich ein Gesamtbudget von 6.300 EUR ergibt (s. Tabelle~\ref{tab:Kostenaufstellung}).
\tabelle{Kostenaufstellung}{tab:Kostenaufstellung}{Kostenaufstellung}\\

Kosten für die genutzte Hardware werden nicht angesetzt, da davon ausgegangen wird, dass diese schon vorhanden ist.

\subsubsection{Amortisationsdauer}
\label{sec:Amortisationsdauer}
Die Amortisation des Projekts ist nicht das primäre Ziel. Wie oben genannt dient es zur Analyse von Daten in Bezug auf das Nutzungsverhalten der Kunden und zukünftige Optimierungen. Aus diesem Grund ist keine Amortisationsdauer zu berechnen.

Durch die hohe Anzahl an genutzten Diagnostikgeräten von {\betriebNameKzf} in Laboren auf der ganzen Welt, ist der Gewinnverlust hochgerechnet auf ein Gerät sehr gering. Bei einer Anzahl von beispielsweise 5.000 verkauften Geräten beträgt der Gewinnverlust ca. 1,25 EUR pro Gerät.

\subsection{Qualitätsanforderungen}
\label{sec:Qualitätsanforderungen}
Da die Anwendung unter Anderem die Kosten für Mitarbeiterschulungen senken soll, muss sie einfach und intuitiv zu bedienen sein. Bei geringfügigen Fehlern soll die Software ausfallsicher sein und durch die Abfrage von unbestimmten Größen an Datenmengen soll die Kommunikation performant und schlank sein.

Die Bereitstellung einer frei implementierbaren Schnittstelle verlangt eine gute Dokumentation und Kommentierung des Quellcodes, sodass Entwickler diese Schnittstelle einfach nutzen können. Da die Softwares zur Steuerung der Diagnostikgeräte für Windows entwickelt werden, ist es nötig die Schnittstelle ebenfalls für Windows zu entwickeln und daraus resultierend auch die Software zur Abfrage der Betriebsmittelverbräuche.